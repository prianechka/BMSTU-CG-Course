\newpage
\addcontentsline{toc}{chapter}{Введение}

\chapter*{Введение}
В компьютерной графике большое внимание уделяется алгоритмам получения реалистических изображений. 
Они должны учитывать физические явления: преломление, отражение.
При этом с повышением сложности алгоритмов растёт и их требовательность к системным ресурсам. 

Актуальность работы обусловлена необходимостью создания реалистических изображений аппаратными методами, используя оптимальные для этой задачи алгоритмы.

Цель курсовой работы: разработать программу для построения реалистического изображения из перечня геометрических объектов: куб, конус, цилиндр и сфера.

Для достижение поставленное цели, требуется выполнить следующие задачи:
\begin{enumerate}
	\item Провести анализ существующих алгоритмов и выбрать оптимальные пути для решения основной задачи.
	\item Выбрать подходящий способ декомпозиции программы.
	\item Выбрать подходящий язык программирования и среду разработки для выполнения работы.
	\item Создать программный продукт для решения задачи, реализовать выбранные алгоритмы.
	\item Реализовать понятный интерфейс для клиента.
	\item Провести исследования на основе полученных результатов.
\end{enumerate}
